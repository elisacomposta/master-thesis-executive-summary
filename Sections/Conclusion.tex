\section{Conclusions}
\label{sec:conclusion}

% Intro
This work explored the use of LLMs as agents in social simulations.
The original \textit{Y} simulator was extended to integrate opinion modeling, misinformation agents, and a realistic initialization based on real-world data.

% Main results
The simulations showed that LLM-based agents are capable of interacting, generating content with different linguistic styles, and forming social connections. 
Opinion scores assigned by LLMs followed trends similar to those of traditional opinion dynamics models, supporting their validity for population-level analysis.

% Limitations
However, some limitations emerged.
First, the duration of the simulations (21 virtual days) was not sufficient to observe long-term effects.
For example, the impact of different content recommendation systems didn't emerge, as the network was not yet sufficiently structured in the first virtual days. 
Another limitation is that agents, despite being enriched with personality traits and confirmation bias, lacked some cognitive and emotional mechanisms necessary to reproduce real-world dynamics, such as the susceptibility to misinformation, which resulted negligible in the simulations.

% Future work
Future research could explore a wider range of disinformation strategies (e.g., bots, coordinated groups), integrate multimodal content (text, images, videos), or introduce external events during the simulation timeline. 
Additionally, a comparison of the simulation results with real-world data would help assess the realism of the emergent behaviors and validate LLM-based agents.

% Conclusion
Overall, this work shows that LLMs are a powerful tool for simulating complex social phenomena, enabling more realistic modeling of language and interaction in agent-based systems.
However, replicating complex human traits, such as emotional reasoning and susceptibility to manipulation, remains a challenge.
Further studies are required to make these simulations closer to real-world dynamics by capturing deeper individual and social factors.
