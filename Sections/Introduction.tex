\section{Introduction}
\label{sec:introduction}

Online social networks are a central aspect of the daily life of millions of people. They are not just communication platforms but digital spaces where users express their emotions and shape opinions.
For this reason, they offer a valuable opportunity for studying social dynamics.
This is the main focus of Computational Social Science, interdisciplinary field that uses computational methods to study and explain human behavior and social interactions.

To study these social processes in controlled conditions, simulators are used to recreate virtual environments.
Among these, Agent-Based Modeling (ABM) simulates social systems by defining individual agents that follow simple rules, and whose interactions lead to collective behaviors.
However, traditional ABMs struggle to capture the full complexity of human behavior, such as language tone and emotions.

Large Language Models (LLMs) offer an improvement by enabling agents that simulate conversations in natural language, and express opinions and emotions.

A particularly important aspect of online social media is the spread of misinformation and disinformation, which can influence opinions and social dynamics. 
Simulating these phenomena can help us better understand their impact and explore possible mitigation strategies.

% Contribution
This work explores the behavior of LLM-based agents in a simulated social network, by extending an existing simulator into three main directions:
(i) integrating real-world data for initializing agents to improve the realism of their behavior;
(ii) introducing an explicit opinion model that allows opinions to evolve over time;
(iii) defining a new category of agents that generate misleading content to support their views.


% Research questions
Based on this framework, the study addresses the following research questions:
\begin{itemize}
    \item Can LLM-based agents realistically simulate social dynamics in online platforms, including phenomena such as opinion evolution, misinformation diffusion, and network formation?
    \item What's the impact of misinformation on opinion shift?
    \item What are the current limitations of using LLM agents?
\end{itemize}

By addressing these questions, this work aims to evaluate the potential of LLMs as social agents and identify the key challenges of using them to model social dynamics.
