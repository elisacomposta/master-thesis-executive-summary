\section{Methods}
\label{sec:methods}

%% Simulation workflow
\subsection{Simulation workflow}
The results discussed in this work are based on \textit{Y} \cite{rossetti2024ysocialllmpoweredsocial}, social media simulator with LLM agents. 

Each simulated day consists of multiple rounds during which a sample of active agents performs actions, such as posting and reacting to content.
Agents start with no predefined social connections, allowing the network structure to emerge and evolve over time based on their interactions. 
The system is highly configurable, as it allows to specify parameters such as hourly activity, recommendation algorithms, and misinformation level.
In addition to the existing simulator, at the end of each day an additional phase enables agents to update their opinions on the discussed topics.


%% Agents
\subsection{Agents}

This section explores how agents are modeled, one the most crucial aspects in social simulations.

% Init
\subsubsection{Initialization and behavior}
When creating the population, each agent receives a detailed profile built from a mix of randomly generated features, real-world data, and information sampled from real-world distributions.

The randomly sampled dimensions include name, surname, and personality, which follows the Big Five model, allowing up to 32 combinations of distinct personalities.
Age and gender are assigned using weighted probabilities based on 2024 Twitter statistics in Italy, restricted to users aged 18-60 \cite{statista2024twitter}.

All agents are set with Italian nationality and share four main interests, corresponding to the political topics analyzed in this study: \textit{Civil rights}, \textit{Immigration}, \textit{Nuclear energy}, \textit{Reddito di Cittadinanza}.

A Twitter dataset collected around the 2022 Italian political elections \cite{pierri2023ita} has been used to initialize some features: supported political leaning, writing toxicity, and activity level, normalized in the range $[0,1]$ using a logarithmic transformation.

Before performing an action, each agent receives a role prompt describing its persona, its opinions, and a description of the topics, ensuring it has the necessary background knowledge and stance definition.


% Behavior
\medskip
When an agent is active, it performs one of the following actions: \textit{post}, \textit{comment} on an existing conversation, or just \textit{read} a tweet.
The action is selected based on two activity values (in $[0,1]$) that define how likely the agent it to post or comment.
If their sum is less then 1, the remaining probability is automatically assigned to the read action.
This setup was designed to leverage the available data of real-world user activities.

The topic to be discussed in a post is randomly picked from the user's interests, among those active in the configured time window.
Although the agents represent Italian users and topics, all generated content in the simulation is in English, reflecting the default language setting of the simulation environment.
When the agent comments or reads a post, it can also decide to add a reaction (\textit{like} or \textit{dislike}) and to \textit{follow} or \textit{unfollow} the author.
These additional behaviors contribute to shaping the network over time.

To decide which content the agent interacts with, a recommendation system selects the posts to show. Two algorithms are used: \textit{ReverseChronoFollowersPopularity}, that recommends popular recent content mainly from followed users, and \textit{ContentRecSys}, selecting random posts.
For suggesting users to follow, the default algorithm, \textit{PreferentialAttachment}, is used, which ranks users according to the product of the agent’s neighbor set size and that of the candidate user.


% Misinfo
\subsubsection{Misinformation agents}
This work introduces a new agent type that generates misleading content.
These are designed like normal users, but are prompted to spread misinformation.
They act individually, without coordination or harmful intent, and are therefore considered as misinformation agents.

Unlike other individuals, they are not initialized using directly real user data.
Their political leaning is assigned ensuring a uniform distribution across coalitions.
Toxicity and activity levels are generated using statistical distributions fitted on the dataset: for toxicity the best-fitting distribution is identified per coalition; for activity, posts and comments are modeled with discrete distributions and then converted into normalized scores, consistently with other agents.


%% Opinion
\subsection{Opinion modeling and update}
To better represent individual behavior, this work introduces an explicit opinion model.
Opinions are represented by a numerical score from -1 (strongly opposed) to +1 (strongly supported), along with a textual explanation. 
The initial opinion of each agent reflects the stance of their supported political coalition, and can evolve during the simulation, using either a mathematical model or an LLM-based approach.

\medskip
Among the mathematical approaches, the implementation includes the classical Friedkin-Johnsen model, where agents update their opinion by combining their initial stance with those of their neighbors.
The model used throughout the simulation is based on \cite{Ye2018Opinion}, and updates opinions based on the agent's current state rather than the initial one:
\[
x_i(t + 1) = (1 - \lambda_i) x_i(t) + \lambda_i  \sum_{j \in N_i(t)} w_{ij} x_j (t)
\]
where $x_i(t)$ is the opinion of individual $i$ at time $t$, $N_i$ is the set of following users, $\lambda_i$ is the user's susceptibility to other users, and $w_{ij}$ is the impact user $j$ has on $i$. 
Susceptibility is computed from personality traits, and the neighbors are weighted according to the type of interactions they had (\textit{like}, \textit{dislike}, \textit{follow}). 

These scores are only used for analysis, as all actions in the simulations are entirely driven by the LLM agents.


% LLM-based opinion update
\medskip
In the LLM-based opinion model, LLMs use the textual description of their opinions to act coherently with their views during the simulations.

In the update phase, the agent receives its own profile, the topics to update, its current opinions and the beliefs of its coalition, and a memory of recent interactions, including the content read/written, reactions, follow changes.
The model then outputs both a textual explanation and a stance label, later mapped to a numerical score.

To simulate a slight resistance to change, the prompt also includes a confirmation bias, which is stronger for misinformation agents, to reflect their stronger attachment to their beliefs.

