\section{Methods}
\label{sec:methods}

%% Simulation workflow
\subsection{Simulation workflow}
The simulations discussed in this work are based on the Y, framework designed to realistically replicate a social media platform with virtual agents. 

Each simulated day consists of multiple rounds during which a sample of active agents performs actions such as posting and interacting.
Agents start without predefined social connections, allowing the network structure to emerge and evolve over time based on their interactions. 
The system is highly configurable, as it allows to specify parameters such as hourly activity, recommendation algorithms, and agents’ misinformation levels.
In addition to the existing simulator, at the end of each day an additional phase enables agents to update their opinions on the discussed topics.


%% Agents
\subsection{Agents}

One of the biggest challenges in social simulations is to realistically model agents and their behavior. 
This section describes how agents are initialized and modeled, and introduces a new category of agents diffusing misinformation.

% Init
\subsubsection{Initialization}
When creating the population, each agent receives a detailed profile build from a mix of randomly generated features, real-world data, and data sampled from real-world distributions.

Some profile dimensions are randomly sampled: name, surname, email, password and personality.
Personality follows the Big Five model, allowing up to 32 combinations of distinct personalities.
Age and gender are assigned using weighted probabilities based on 2024 Twitter statistics in Italy, restricted to users aged 18-60.

All agents are set with Italian nationality and share four main interests, corresponding to the political topics analyzed in this study: \textit{Civil rights}, \textit{Immigration}, \textit{Nuclear energy}, \textit{Reddito di Cittadinanza}.

To make the users even more realistic and context-aware, some attributes are initialized using real-world data, from a Twitter dataset collected around the 2022 Italian political elections \cite{pierri2023ita}.
This includes the users' political leaning, writing toxicity and activity level.
The activity is normalized in the range $[0,1]$ using a logarithmic transformation to reduce the effect of outliers.

Agents are orchestrated using AutoGen, which enables multi-agent conversations.
Before performing any action, the agents receive a role prompt, specifying its profile, the personal opinions, the supported coalition and their views, and a description of the topics.
The topic descriptions ensure that the agents have the necessary background knowledge on the considered context, and specify the meaning of the stances for each topic.


% Behavior
\subsubsection{Agent's behavior}
When an agent is active, it performs one of the following actions: post a tweet, comment on an existing conversation, or just read a tweet.
The action is selected based on two activity values (in $[0,1]$) that define how likely the agent it to post or comment.
If their sum is less then 1, the remaining probability is automatically assigned to the read action.
This setup helps model user behavior more realistically, using values based on real data.

If the agents writes a post, the topic is randomly picked from the user's interests, among those active in the configured time window.
When the agent comments or reads a post, it can also decide to add a reaction (\textit{like} or \textit{dislike}) and to follow or unfollow the author.
These additional behaviors contribute to shaping the network over time.

To decide which content the agent interacts with, a recommendation system selects the posts to show. Two algorithms are used: \textit{ReverseChronoFollowersPopularity}, that recommends popular recent content mainly from followed users, and \textit{ContentRecSys}, selecting random posts.
For suggesting users to follow, the system \textit{PreferentialAttachment} default algorithm is used, which ranks users according to the product of the agent’s neighbor set size and that of the candidate user.


% Misinfo
\subsubsection{Misinformation Agents}


%% Opinion
\subsection{Opinion modeling and update}